\documentclass[iop]{emulateapj}

%\usepackage[english]{babel}
\usepackage[utf8x]{inputenc}
\usepackage{amsmath}
\usepackage{graphicx}
%\usepackage[colorinlistoftodos]{todonotes}




\shortauthors{People et al.}
\shorttitle{Indirect imaging of a star using planet transits}
\begin{document}
\bibliographystyle{apj}
\title{Indirect imaging of a star using planet transits}
\author{
BHF-M
}


\begin{abstract}
We use transits to measure limb darkening. We picked a star and fit kick-ass models to the transits. Theory says this but we get this other thing. Using analytical limb darkened code we get X but with a non-parametric model we get Y. Transits and learning things. 
\end{abstract}

\section{Introduction}
A single transit of an exoplanet is a measurement of the brightness profile across on one cord of a star. 

Talk about star spots, limb darkening, information from interferometry. 

Talk about theory, different model limb darkening, limitations.

Power of multi planet systems.


\section{Transit modeling}
Introduce the models -- BART and Mandel and Agol.

\section{Applying the model}
Fitting the models to the data of KOI-3158.



\section{Constraints on limb darkening}
Cool stuff we do

\section{Conclusions}
We are awesome


\end{document}
